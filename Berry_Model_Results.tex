\documentclass[11pt]{article}
\usepackage[margin=1in]{geometry}
\usepackage{amsmath}
\usepackage{booktabs}
\usepackage{lmodern}
\usepackage[T1]{fontenc}
\usepackage{setspace}

\title{Berry Airline Entry Model: Estimation Results}
\author{Vatsal Mitesh Tailor}
\date{\today}

\begin{document}

\maketitle
\onehalfspacing

\section{Introduction}

This note estimates a static entry model for US airline markets using the
Berry (1992) framework. The data cover 2{,}742 origin--destination markets with
up to six potential carriers. Firm profits are
\begin{equation}
  \pi_{im} = V_{im} - \delta N_{-im} + \varepsilon_{im},
\end{equation}
where $V_{im}$ is a profit index from market covariates, $N_{-im}$ is the
number of rivals, $\delta > 0$ is the competition effect, and $\varepsilon_{im}
\sim N(0,1)$. We observe entry decisions but not profit shocks.

\section{Model specifications}

\subsection{Symmetric-firm specification}

All firms are treated as identical. The profit index is
\begin{equation}
  V_{im} = X_m'\beta, \quad X_m = (1, \log(\text{pop}_m), \log(\text{dist}_m),
  \log(\text{dist}_m^2)),
\end{equation}
with market size, distance, and distance squared. Firms differ only through
$\varepsilon_{im}$. Parameters are estimated by maximum likelihood.

\subsection{Heterogeneous-firm specification}

This specification allows firm-specific effects. The profit index is
\begin{equation}
  V_{im} = X_m'\beta + s_i,
\end{equation}
where $X_m$ includes intercept, average log population, tourism, log distance,
and log passengers. The $s_i$ are airline-specific size parameters. Entry
probabilities use a sequential rule where the highest-profit firm enters first.

\section{Results}

Table~\ref{tab:symmetric} shows estimates for the symmetric specification.

\begin{table}[h!]
  \centering
  \caption{Symmetric-firm specification}
  \label{tab:symmetric}
  \begin{tabular}{l r}
    \toprule
    Parameter & Estimate \\ \midrule
    Constant & 1.296 \\
    $\log(\text{population})$ & $-0.078$ \\
    $\log(\text{distance})$ & 0.096 \\
    $\log(\text{distance}^2)$ & $-0.095$ \\
    Competition effect $\delta$ & 5.032 \\ \bottomrule
  \end{tabular}
\end{table}

The competition parameter $\delta = 5.03$ is large, meaning additional entrants
reduce profits substantially. Log-likelihood is $-2318.47$ and prediction
accuracy is 62\,\%.

Table~\ref{tab:heterogeneous} shows the heterogeneous specification.

\begin{table}[h!]
  \centering
  \caption{Heterogeneous-firm specification}
  \label{tab:heterogeneous}
  \begin{tabular}{l r}
    \toprule
    Parameter & Estimate \\ \midrule
    Constant & 0.329 \\
    Average log population & $-0.126$ \\
    Tourism & $-0.087$ \\
    $\log(\text{distance})$ & 0.287 \\
    $\log(\text{passengers})$ & 0.840 \\
    Competition effect $\delta$ & 1.481 \\ \bottomrule
  \end{tabular}
\end{table}

Log passengers has a large positive coefficient (0.84), consistent with demand
driving entry. The competition parameter $\delta = 1.48$ is smaller than in the
symmetric model. Log-likelihood is $-1908.39$ and accuracy is 64\,\%.

\section{Comparison}

Table~\ref{tab:comparison} compares the two models. The heterogeneous
specification fits better with a higher log-likelihood and accuracy. Adding
airline-specific effects improves the model.

\begin{table}[h!]
  \centering
  \caption{Model comparison}
  \label{tab:comparison}
  \begin{tabular}{l r r r}
    \toprule
    Specification & Log-likelihood & $\delta$ & Accuracy (\%) \\ \midrule
    Symmetric firms & $-2318.47$ & 5.03 & 62.1 \\
    Heterogeneous firms & $-1908.39$ & 1.48 & 64.3 \\ \bottomrule
  \end{tabular}
\end{table}

Both models show that competition reduces profits ($\delta > 0$). The symmetric
model has a larger $\delta$ to compensate for ignoring firm differences. Once
we add airline-specific size terms, a smaller $\delta$ is sufficient to match
the data.

\begin{thebibliography}{9}

\bibitem{berry1992}
Berry, S. T. (1992).
``Estimation of a model of entry in the airline industry.''
\emph{Econometrica}, 60(4), 889--917.

\end{thebibliography}

\end{document}
