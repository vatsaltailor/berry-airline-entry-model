\documentclass[11pt]{article}
\usepackage[margin=1in]{geometry}
\usepackage{amsmath}
\usepackage{amssymb}
\usepackage{booktabs}
\usepackage{graphicx}
\usepackage{hyperref}

\title{\textbf{Estimation of Airline Entry Model} \\ \large{Implementation of Berry (1992) Static Entry Game}}
\author{Vatsal Mitesh Tailor}
\date{\today}

\begin{document}

\maketitle

\section{Introduction and Methodology}

This paper implements the static entry game model developed by Berry (1992) to analyze airline entry decisions across 2,742 U.S. city-pair markets. The model estimates how market fundamentals and competitive effects influence airlines' decisions to enter or stay out of specific routes.

\subsection{Theoretical Framework}

The model assumes that airline $i$ enters market $m$ if its expected profit is positive:
\begin{equation}
\pi_{i,m} = V_{i,m} - \delta \cdot N_{-i,m} + \varepsilon_{i,m} > 0
\end{equation}

where $V_{i,m}$ is the deterministic profit component, $\delta$ captures the competitive effect (profit reduction per competitor), $N_{-i,m}$ is the number of other firms entering, and $\varepsilon_{i,m} \sim N(0,1)$ is an idiosyncratic shock.

The deterministic profit is specified as:
\begin{equation}
V_{i,m} = \alpha + \beta_1 \text{MarketSize}_m + \beta_2 \text{Tourism}_m + \beta_3 \text{Distance}_m + \beta_4 \text{Passengers}_m + \text{AirlineSize}_i
\end{equation}

We observe three possible outcomes in each market: no entry ($N=0$), monopoly ($N=1$), or competition ($N \geq 2$). Following Berry's sequential entry assumption, firms with higher expected profits move first. This allows us to calculate the probability of each outcome using the normal CDF.

\subsection{Estimation Approach}

Parameters are estimated via Maximum Likelihood Estimation (MLE). For each market, we calculate:
\begin{align}
P(N=0) &= \prod_{i=1}^{6} \Phi(-V_{i,m}) \\
P(N=1) &= \Phi(V_{\max}) \prod_{j \neq \max} \Phi(-(V_{j,m} - \delta)) \\
P(N \geq 2) &= 1 - P(N=0) - P(N=1)
\end{align}

where $\Phi(\cdot)$ is the standard normal CDF and $V_{\max}$ is the highest profit among all firms. The log-likelihood function is:
\begin{equation}
\mathcal{L}(\theta) = \sum_{m=1}^{2742} \left[ \mathbb{1}_{N_m=0} \log P(N=0) + \mathbb{1}_{N_m=1} \log P(N=1) + \mathbb{1}_{N_m \geq 2} \log P(N \geq 2) \right]
\end{equation}

We maximize this likelihood using the L-BFGS-B algorithm, which handles the non-linear optimization efficiently.

\section{Results and Interpretation}

\subsection{Parameter Estimates}

Table \ref{tab:results} presents the estimated parameters. The model successfully converged with a log-likelihood of $-1908.39$.

\begin{table}[h]
\centering
\caption{Parameter Estimates}
\label{tab:results}
\begin{tabular}{lc}
\toprule
\textbf{Parameter} & \textbf{Estimate} \\
\midrule
Constant ($\alpha$) & 0.3293 \\
Market Size ($\beta_1$) & $-0.1258$ \\
Tourism ($\beta_2$) & $-0.0873$ \\
Distance ($\beta_3$) & 0.2868 \\
Passengers ($\beta_4$) & 0.8397 \\
Competition Effect ($\delta$) & 1.4809 \\
\bottomrule
\end{tabular}
\end{table}

\subsection{Economic Interpretation}

\textbf{Passenger Volume ($\beta_4 = 0.84$):} The strongest predictor of entry. Markets with higher passenger traffic are significantly more profitable, making them attractive for airline entry. This coefficient is economically significant and aligns with industry practice.

\textbf{Distance ($\beta_3 = 0.29$):} Longer routes are more profitable, consistent with airlines earning higher margins on long-haul flights due to pricing power and operational efficiencies.

\textbf{Market Size ($\beta_1 = -0.13$):} After controlling for passenger volume, larger population markets show slightly lower profitability. This may reflect higher operating costs, more intense competition, or regulatory constraints in major metropolitan areas.

\textbf{Competition Effect ($\delta = 1.48$):} Each additional competitor reduces profit by 1.48 units. This substantial competitive effect explains the market structure observed in the data: 62\% of markets have 2+ airlines, 31\% are monopolies, and only 7\% have no service. The high $\delta$ creates natural entry barriers in smaller markets.

\subsection{Model Fit and Validation}

The data contains 200 markets with no entry, 840 monopoly markets, and 1,702 competitive markets. The model successfully captures this distribution through the estimated parameters. The log-likelihood of $-1908.39$ indicates reasonable fit given the discrete nature of outcomes and market heterogeneity.

The results are economically sensible: passenger demand drives entry, competition reduces profitability, and longer routes are more attractive. These findings align with theoretical predictions and empirical patterns in airline markets.

\section{Conclusion}

This implementation successfully estimates the Berry (1992) static entry game model for airline markets. The results demonstrate that entry decisions are primarily driven by passenger volume and competitive effects. The substantial competition parameter ($\delta = 1.48$) explains why many markets support only one or two carriers despite positive base profitability.

The model provides insights into market structure determination and could be extended to analyze counterfactual scenarios such as merger effects or entry subsidies. Future work could incorporate dynamic considerations or allow for heterogeneous competitive effects across airline types.

\end{document}
